\nonstopmode{}
\documentclass[letterpaper]{book}
\usepackage[times,hyper]{Rd}
\usepackage{makeidx}
\usepackage[utf8]{inputenc} % @SET ENCODING@
% \usepackage{graphicx} % @USE GRAPHICX@
\makeindex{}
\begin{document}
\chapter*{}
\begin{center}
{\textbf{\huge Package `agghoo'}}
\par\bigskip{\large \today}
\end{center}
\ifthenelse{\boolean{Rd@use@hyper}}{\hypersetup{pdftitle = {agghoo: Aggregated Hold-Out Cross Validation}}}{}
\begin{description}
\raggedright{}
\item[Title]\AsIs{Aggregated Hold-Out Cross Validation}
\item[Date]\AsIs{2022-08-30}
\item[Version]\AsIs{0.1-0}
\item[Description]\AsIs{The 'agghoo' procedure is an alternative to usual cross-validation.
Instead of choosing the best model trained on V subsamples, it determines
a winner model for each subsample, and then aggregate the V outputs.
For the details, see ``Aggregated hold-out'' by Guillaume Maillard,
Sylvain Arlot, Matthieu Lerasle (2021) <}\Rhref{https://arxiv.org/abs/1909.04890}{arXiv:1909.04890}\AsIs{>
published in Journal of Machine Learning Research 22(20):1--55.}
\item[Author]\AsIs{Sylvain Arlot }\email{sylvain.arlot@universite-paris-saclay.fr}\AsIs{ [cph,ctb],
Benjamin Auder }\email{benjamin.auder@universite-paris-saclay.fr}\AsIs{ [aut,cre,cph],
Melina Gallopin }\email{melina.gallopin@universite-paris-saclay.fr}\AsIs{ [cph,ctb],
Matthieu Lerasle }\email{matthieu.lerasle@universite-paris-saclay.fr}\AsIs{ [cph,ctb],
Guillaume Maillard }\email{guillaume.maillard@uni.lu}\AsIs{ [cph,ctb]}
\item[Maintainer]\AsIs{Benjamin Auder }\email{benjamin.auder@universite-paris-saclay.fr}\AsIs{}
\item[Depends]\AsIs{R (>= 3.5.0)}
\item[Imports]\AsIs{class, parallel, R6, rpart, FNN}
\item[Suggests]\AsIs{roxygen2}
\item[URL]\AsIs{}\url{https://git.auder.net/?p=agghoo.git}\AsIs{}
\item[License]\AsIs{MIT + file LICENSE}
\item[RoxygenNote]\AsIs{7.2.1}
\item[Collate]\AsIs{'compareTo.R' 'agghoo.R' 'R6_AgghooCV.R' 'R6_Model.R'
'checks.R' 'utils.R' 'A_NAMESPACE.R'}
\item[NeedsCompilation]\AsIs{no}
\end{description}
\Rdcontents{\R{} topics documented:}
\printindex{}
\end{document}
